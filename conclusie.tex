%%=============================================================================
%% Conclusie
%%=============================================================================

\chapter{Conclusie}
\label{ch:conclusie}

%% TODO: Trek een duidelijke conclusie, in de vorm van een antwoord op de
%% onderzoeksvra(a)g(en). Wat was jouw bijdrage aan het onderzoeksdomein en
%% hoe biedt dit meerwaarde aan het vakgebied/doelgroep? Reflecteer kritisch
%% over het resultaat. Had je deze uitkomst verwacht? Zijn er zaken die nog
%% niet duidelijk zijn? Heeft het onderzoek geleid tot nieuwe vragen die
%% uitnodigen tot verder onderzoek?

%\lipsum[76-80]

Uit dit onderzoek kan vastgesteld worden dat Custom Vision het meest betrouwbare framework is om een eigen afbeeldingclassificatie model te ontwikkelen. Het is verder sterk genoeg om producten die nauwelijks verschillen uit elkaar te houden. Bovendien is dit framework veel eenvoudiger in gebruik dan Tensorflow of Turi Create. Ook ontwikkelaars die nauwelijks kennis hebben van machine learning kunnen dus eenvoudig een product herkennend model creëren met behulp van Custom Vision. 

Een fysieke winkel kan een innovatieve meerwaarde ontwikkelen door een applicatie op de markt te brengen die producten uit de handelszaak herkend. Daarnaast kan de winkel met behulp van de applicatie veel meer productinformatie aanbieden aan de klant. Bovendien kunnen er meerdere verkooptechnieken worden toegepast, zo heeft het presenteren van bijhorende producten gunstige invloed op het koopgedrag van de consument.  Het is ook interessant om alternatieven voor te leggen via de applicatie. 

Dit onderzoek kan een bijdrage leveren aan zaakvoerders die overwegen om een product herkennende applicatie te laten ontwikkelen. Het kan ook een meerwaarde bieden tot andere onderzoeken omtrent machine learning en afbeeldingclassificatie. Een uitbreiding op dit werkstuk zou een onderzoek zijn naar productherkenning voor voedingsmiddelen in een supermarkt. 