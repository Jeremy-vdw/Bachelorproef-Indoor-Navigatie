%%=============================================================================
%% Samenvatting
%%=============================================================================

% TODO: De "abstract" of samenvatting is een kernachtige (~ 1 blz. voor een
% thesis) synthese van het document.
%
% Deze aspecten moeten zeker aan bod komen:
% - Context: waarom is dit werk belangrijk?
% - Nood: waarom moest dit onderzocht worden?
% - Taak: wat heb je precies gedaan?
% - Object: wat staat in dit document geschreven?
% - Resultaat: wat was het resultaat?
% - Conclusie: wat is/zijn de belangrijkste conclusie(s)?
% - Perspectief: blijven er nog vragen open die in de toekomst nog kunnen
%    onderzocht worden? Wat is een mogelijk vervolg voor jouw onderzoek?
%
% LET OP! Een samenvatting is GEEN voorwoord!

%%---------- Nederlandse samenvatting -----------------------------------------
%
% TODO: Als je je bachelorproef in het Engels schrijft, moet je eerst een
% Nederlandse samenvatting invoegen. Haal daarvoor onderstaande code uit
% commentaar.
% Wie zijn bachelorproef in het Nederlands schrijft, kan dit negeren, de inhoud
% wordt niet in het document ingevoegd.

\IfLanguageName{english}{%
\selectlanguage{dutch}
\chapter*{Samenvatting}
\lipsum[1-4]
\selectlanguage{english}
}{}

%%---------- Samenvatting -----------------------------------------------------
% De samenvatting in de hoofdtaal van het document

\chapter*{\IfLanguageName{dutch}{Samenvatting}{Abstract}}

Op het internet is er een enorm aanbod van machine learning frameworks die tools hebben om een machine te ontwikkelen die afbeeldingen classificeert op basis van een dataset. Maar daar kwam vaak veel vakkennis bij kijken. Afgelopen jaar zijn echter twee nieuwe frameworks op de markt gekomen waarvoor heel wat minder kennis nodig is. In deze bachelorproef werd onderzocht of een eenvoudig toepasbaar framework even nauwkeurig is als een klassiek framework in het classificeren van afbeeldingen. Dit kan ontwikkelaars helpen bij hun keuze voor een framework om afbeeldingclassificatie mogelijk te maken met hun dataset. 

Aanvullend deed deze scriptie ook onderzoek op commercieel vlak. Aangezien het aantal online webwinkels blijft stijgen, moeten bestaande fysieke winkels innovatief groeien om te kunnen concurreren. Daarom werd onderzocht of een applicatie die producten uit een bepaalde handelszaak herkend een toegevoegde waarde biedt aan desbetreffende winkel. Bovendien werd het koopgedrag van de consument onderzocht met behulp van de applicatie. Dit kan zaakvoerders helpen beslissen in het investeren van nieuwe technologieën. 

Vooreest werd via literatuurstudie nagegaan hoe een framework erin slaagde om een aangepast afbeeldingsclassificatie model te ontwikkelen. Daarnaast is er met behulp van de literatuurstudie gekozen voor Tensorflow als klassiek framework. Dit werd vergeleken met eenvoudige nieuwe frameworks, Turi Create en Microsoft Custom Vision. Voor verder onderzoek werd met behulp van een demoapplicatie een enquête gehouden bij ervaren smartphone gebruikers in een handelszaak. 

Uit onderzoek bleek dat een afbeeldingsclassificatie model ontwikkelen via het framework van Microsoft de meest nauwkeurige resultaten opleverde. Bovendien is opgenoemd framework ook het eenvoudigste toepasbaar voor ontwikkelaars. Uit verdere resultaten kon afgeleid worden dat heel wat consumenten ervan overtuigd zijn dat een product herkennende applicatie een meerwaarde biedt aan de zaak. Daarenboven zou een klant meer aankopen dan voorzien met betreffende applicatie. 

