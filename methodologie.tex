%%=============================================================================
%% Methodologie
%%=============================================================================

\chapter{Methodologie}
\label{ch:methodologie}

%% TODO: Hoe ben je te werk gegaan? Verdeel je onderzoek in grote fasen, en
%% licht in elke fase toe welke stappen je gevolgd hebt. Verantwoord waarom je
%% op deze manier te werk gegaan bent. Je moet kunnen aantonen dat je de best
%% mogelijke manier toegepast hebt om een antwoord te vinden op de
%% onderzoeksvraag.

\section{Opbouw van het onderzoek}
\label{sec:Opbouw van het onderzoek}

Dit onderzoek vergelijkt drie frameworks voor afbeeldingclassificatie. Er wordt onderzocht of eenvoudig implementeerbare frameworks even nauwkeurig resulteren als complexere frameworks. De drie frameworks worden getraind met negen producten uit de Feestwinkel, te Oudenaarde. De producten worden onderverdeeld in drie groepen: kleine producten, middelgrote producten en grote producten. Per groep wordt er voor elk framework een eigen classificatiemodel gecreëerd.

Als eerste framework is er gekozen voor een standaard machine learning bibliotheek. Onder deze noemer vallen frameworks zoals Caffe, Torch, Theano en Tensorflow. In dit onderzoek is gekozen voor Tensorflow, het scoort naast Theano het beste rekening houdend met snelheid, classificatie nauwkeurigheid en complexiteit van implementatie \autocite{researchgate}. Bovendien is er heel wat informatie te vinden op het web omtrent het toepassen van Tensorflow voor verschillende doeleinden \autocite{huyen}. De andere twee frameworks die worden toegepast in dit onderzoek zijn Turi Create en Microsoft Custom Vision. Deze frameworks zijn respectievelijk eenvoudiger in het ontwikkelen van een eigen afbeeldingclassificatie model. De drie frameworks zijn gratis te gebruiken. 

Nadat een eigen classificatie model getraind is met de trainingsdata per groep van producten, zal het model geëxporteerd worden naar een CoreML model. Het CoreML framework is gelanceerd door Apple op het WWDC2017. Het is een nieuwe toepassing in iOS 11 en het zorgt ervoor dat ontwikkelaars gemakkelijk machine learning models kunnen uitvoeren in hun applicaties. Zo ondersteund ook CoreML functies zoals gezichts- herkenning en tekstherkenning \autocite{coreml}. 

\section{Frameworks}
\label{sec:Frameworks}

\subsection{Tensorflow}
\label{ssec:Tensorflow}

Tensorflow is open source deep learning framework door Google. Het is oorspronkelijk ontwikkeld in 2015 door het Google Brain Team om onderzoek te doen naar zowel machine learning, deep learning en neurale netwerken. In 2017 is de eerste open source versie van Tensorflow uitgebracht. Tensorflow wordt onder andere gebruikt voor tekstherkenning en stemherkenning maar ook voor afbeeldingclassificatie. De gemakkelijkste en de meest complete API van dit framework is de Python API, maar Tensorflow levert ook nog een API’s in Java, C++ en Go \autocite{introTensorflow}.

Dit framework kan ook worden toegepast om een eigen afbeeldingclassificatie model te ontwikkelen. Het bouwen van een model vanaf nul kan enkele weken duren, daarom gebruikt Tensorflow een deel van een classificatie model die reeds eerder getraind is en behoorlijk presteert in het classificeren van afbeeldingen uit allerhande categorieën. Het model dat benut wordt is Inception v3 en het is in getraind met ImageNet, een grote database van afbeeldingen. Het CNN kan afbeeldingen classificeren in duizend verschillende klassen \autocite{Tensorflow}.

Tensorflow maakt gebruik van transfer learning om een eigen model te generen met eigen trainingsdata. Het is minder efficiënt dan het trainen van het model vanaf nul, maar het maakt het wel mogelijk om een degelijk model te ontwikkelen met aanzienlijk veel minder trainingsdata en tijd. Tensorflow berekend voor elke afbeelding uit de training set de bottleneckwaarde en slaat deze tijdelijk op. De term ‘bottleneck’ wordt door Tensorflow gebruikt als een verwijzing naar de laag net voor de laatste laag die verantwoordelijk is voor de classificatie. Tensorflow verwijderd de laatste laag van het Inception v3 model, berekend de bottleneckwaarde van elke afbeelding, en traint een nieuwe laatste laag voor classificatie. Standaard voert Tensorflow vierduizend trainingstappen uit met tien willekeurige afbeeldingen uit de trainingsdata. De bottleneckwaarden van de afbeeldingen worden opgeslagen zodat de waarden niet steeds opnieuw moeten berekend worden door het model wanneer de afbeeldingen hergebruikt worden in de trainingstap \autocite{deeplearningTensor}.

\subsection{Turi Create}
\label{ssec:Turi Create}

Apple kocht in 2016 het bedrijf Turi over. Kort daarna gaf Apple het framework Turi Create vrij voor het publiek \autocite{9to5mac}. Apple beweert dat men met hun framework geen deskundige in machine learning hoeft te zijn. Een eigen afbeeldingclassificatie model kan gebouwd worden in enkele lijnen Python. Daarnaast kan het framework gebruikt worden voor onder andere objectdetectie en tekstclassificatie \autocite{githubturi}.

Turi Create functioneert volledig gelijklopend zoals Tensorflow. Het past ook transfer learning, met het verschil dat het dit framework gebruik maakt van het convolutional neural network ResNet-50 en niet Inception v3 \autocite{hackermoon}. Uit een vergelijkende studie van de nauwkeurigheid van verschillende CNN’s scoort Inception v3 met een nauwkeurigheid van 79\%, waar ResNet-50 scoort met 76\% \autocite{arxiv}.

\subsection{Microsoft Custom Vision}
\label{ssec:Microsoft Custom Vision}

Microsoft lanceerde de Custom Vision API in 2017. De API is een uitbreiding op de Microsoft Azure Cognitive Services, een verzameling van verschilllende machine learning API’s. Ontwikkelaars kunnen op die manier eenvoudiger functies van artificiële intelligentie toepassen in hun applicaties. Zo kunnen de API’s gebruikt worden voor onder andere gezichts-, en spraakherkenning. Maar ook video- en emotiedetectie. Daarnaast kunnen de services benut worden voor tekstanalyse en taalbegrip \autocite{cognitiveservices}.

De Custom Vision API wordt gebruikt om een eigen afbeeldingclassificatie model te ontwikkelen. Deze API is in de vorm van een webapplicatie en vereist geen enkele kennis omtrent machine learning. Er dienen enkele afbeeldingen te worden online geplaatst per categorie en het model wordt in luttele minuten getraind \autocite{appliedies}.

De details over hoe microsoft erin slaagt om met een ettelijk aantal afbeeldingen een nauwkeurig classificatiemodel te creëren, houden ze voor zich. Al lijkt het er sterk op dat ook dit framework transfer learning toepast een voorgetraind CNN. Microsoft heeft namelijk reeds een model dat gebruikt wordt in de Computer Vision API, het model kan een afbeelding classificeren op basis van meer dan tweeduizend categorieën waarop het model is getraind. Past Microsoft op dit model transfer learning toe zoals Tensorflow en Turi Create, dan is het logisch dat een eigen classificatiemodel goed scoort met een minimum aan afbeeldingen \autocite{praeclarum}.







