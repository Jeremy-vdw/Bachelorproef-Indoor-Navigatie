%%=============================================================================
%% Methodologie
%%=============================================================================

\chapter{Methodologie}
\label{ch:methodologie}

%% TODO: Hoe ben je te werk gegaan? Verdeel je onderzoek in grote fasen, en
%% licht in elke fase toe welke stappen je gevolgd hebt. Verantwoord waarom je
%% op deze manier te werk gegaan bent. Je moet kunnen aantonen dat je de best
%% mogelijke manier toegepast hebt om een antwoord te vinden op de
%% onderzoeksvraag.

\section{Opbouw van het onderzoek}
\label{sec:Opbouw van het onderzoek}

Dit onderzoek vergelijkt drie frameworks voor afbeeldingclassificatie. Er wordt onderzocht of eenvoudig implementeerbare frameworks even nauwkeurig resulteren als complexere frameworks. De drie frameworks worden getraind met negen producten uit de Feestwinkel, te Oudenaarde. De producten worden onderverdeeld in drie groepen: kleine producten, middelgrote producten en grote producten. Per groep wordt er voor elk framework een eigen classificatiemodel gecreëerd.

Als eerste framework is er gekozen voor een standaard machine learning bibliotheek. Onder deze noemer vallen frameworks zoals Caffe, Torch, Theano en Tensorflow. In dit onderzoek is gekozen voor Tensorflow, het scoort naast Theano het beste rekening houdend met snelheid, classificatie nauwkeurigheid en complexiteit van implementatie \autocite{Deeplearning}. Bovendien is er heel wat informatie te vinden op het web omtrent het toepassen van Tensorflow voor verschillende doeleinden \autocite{huyen}. De andere twee frameworks die worden toegepast in dit onderzoek zijn Turi Create en Microsoft Custom Vision. Deze frameworks zijn respectievelijk eenvoudiger in het ontwikkelen van een eigen afbeeldingclassificatie model. De drie frameworks zijn gratis te gebruiken. 

Nadat een eigen classificatie model getraind is met de trainingsdata per groep van producten, zal het model geëxporteerd worden naar een CoreML model. Het CoreML framework is gelanceerd door Apple op het \acrshort{WWDC} 2017. Het is een nieuwe toepassing in iOS 11 en het zorgt ervoor dat ontwikkelaars gemakkelijk machine learning models kunnen uitvoeren in hun applicaties. Zo ondersteund ook CoreML functies zoals gezichts- herkenning en tekstherkenning \autocite{coreml}. 

Vervolgens wordt per reeks van producten de drie bijhorende modellen getest op nauwkeurigheid en vergeleken met elkaar met behulp van een testapplicatie in iOS en testafbeeldingen. Hierdoor kan de eerste deelonderzoeksvraag beantwoord worden.

Na voltooiing van de vergelijking wordt een het best scorende en meest toepasbare framework gebruikt om enkele producten te trainen zodat een demoapplicatie om producten te herkennen ontwikkeld kan worden. De applicatie zal vervolgens uitgeprobeerd worden door klanten die ervaring hebben met een smartphone. Achteraf krijgen de consument enkele vragen met betrekking tot de applicatie zodat ook de laatste deelonderzoeksvragen een antwoord krijgen. 
\section{Frameworks}
\label{sec:Frameworks}

\subsection{Tensorflow}
\label{ssec:Tensorflow}

Tensorflow is open source deep learning framework door Google. Het is oorspronkelijk ontwikkeld in 2015 door het Google Brain Team om onderzoek te doen naar zowel machine learning, deep learning en neurale netwerken. In 2017 is de eerste open source versie van Tensorflow uitgebracht. Tensorflow wordt onder andere gebruikt voor tekstherkenning en stemherkenning maar ook voor afbeeldingclassificatie. De gemakkelijkste en de meest complete \acrshort{API} van dit framework is de Python \acrshort{API}, maar Tensorflow levert ook nog een API’s in Java, C++ en Go \autocite{introTensorflow}.

Om een aangepast model voor afbeeldingclassificatie te ontwikkelen past Tensorflow de transfer learning techniek toe met 'fixed feature extractor'. Het voorgetrainde model dat standaard benut wordt is Inception v3, die op basis van de dataset van ImageNet getraind werd. Tensorflow geeft echter wel de mogelijkheid om ook andere netwerken te hanteren zoals NASNet. In vergelijking met de andere twee frameworks heeft Tensorflow geen ingebouwde functie om het model te exporteren naar het CoreML-formaat \autocite{deeplearningTensor}.

\subsection{Turi Create}
\label{ssec:Turi Create}

Apple kocht in 2016 het bedrijf Turi over. Kort daarna gaf Apple het framework Turi Create vrij voor het publiek \autocite{9to5}. Apple beweert dat men met hun framework geen deskundige in machine learning hoeft te zijn. Een eigen afbeeldingclassificatie model kan gebouwd worden in enkele lijnen Python. Daarnaast kan het framework gebruikt worden voor onder andere objectdetectie en tekstclassificatie \autocite{githubturi}.

Turi Create functioneert volledig gelijklopend zoals Tensorflow. Het past ook transfer learning, met het verschil dat het dit framework gebruik maakt van het convolutional neural network ResNet-50 en niet Inception v3 \autocite{hackermoon}. Uit een vergelijkende studie van de nauwkeurigheid van verschillende \acrshort{CNN}’s scoort Inception v3 met een nauwkeurigheid van 79\%, waar ResNet-50 scoort met 76\% \autocite{arxiv}.

\subsection{Microsoft Custom Vision}
\label{ssec:Microsoft Custom Vision}

Microsoft lanceerde de Custom Vision API in 2017. De \acrshort{API} is een uitbreiding op de Microsoft Azure Cognitive Services, een verzameling van verschilllende machine learning \acrshort{API}’s. Ontwikkelaars kunnen op die manier eenvoudiger functies van artificiële intelligentie toepassen in hun applicaties. Zo kunnen de \acrshort{API}’s gebruikt worden voor onder andere gezichts-, en spraakherkenning. Maar ook video- en emotiedetectie. Daarnaast kunnen de services benut worden voor tekstanalyse en taalbegrip \autocite{cognitiveservices}.

De Custom Vision Service wordt gebruikt voor het eenvoudig trainen of verbeteren van eigen afbeeldingclassificaties. Deze service is in de vorm van een webapplicatie en vereist geen enkele kennis omtrent machine learning. Bovendien is er geen code nodig om een eigen model te ontwikkelen. Er dienen slechts enkele afbeeldingen per klasse online te worden geplaatst en de afbeeldingclassificatie wordt in luttele minuten getraind. Nadat het model getraind is kan het ook eenvoudig geëxporteerd worden naar het CoreML formaat of andere formaten \autocite{appliedies}.

De details over hoe Microsoft erin slaagt om met een ettelijk aantal afbeeldingen een nauwkeurig classificatiemodel te creëren houden ze voor zich. Al worden er in sommige blogpost geschreven dat Custom Vision ook de techniek transfer learning toepassen met behulp van het \acrshort{CNN} ResNet-50. Vooral omdat Microsoft Bing hetzelfde netwerk gebruikt voor afbeeldingherkenning \autocite{praeclarum}.







