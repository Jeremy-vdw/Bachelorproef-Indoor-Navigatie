%%=============================================================================
%% Inleiding
%%=============================================================================

\chapter{Inleiding}
\label{ch:inleiding}

\section{Probleemstelling}
\label{sec:probleemstelling}

Steeds meer handelszaken verdwijnen door online webwinkels, dit merkt u niet alleen in het dagelijkse leven, ook cijfers benadrukken dit. Tussen 2010 en 2015 sloten maar liefst 10.000 fysieke winkels hun deuren door de opkomst van e-commerce \autocite{knack}.
Sterker nog, in 2017 behaald de Belgische onlinesector een record met net iets meer of 10 miljard euro online-uitgaven. Uit hetzelfde onderzoek van BeCommerce wordt duidelijk dat 8,4 miljoen Belgen minstens één aankoop deden online \autocite{becommerce}.

Wellicht kent ook u de voordelen van online shoppen. Er is een duidelijk overzicht van de beschikbare artikelen en er kan gemakkelijk gezocht worden naar het nodige product. Vaak toont de webshop de beschikbare voorraad, een uitgebreide uitleg en hetzelfde artikel in andere kleuren of alternatieven hierop. Bovendien wordt in vele gevallen de bestelling de dag nadien aan huis geleverd. Wanneer een bestelling niet voldoet aan de verwachtingen van de consument, dan kan deze gemakkelijk teruggestuurd worden. 

Ook een webwinkel haalt voordelen uit het online winkelen. Er bestaan heel wat verkooptechnieken die online gemakkelijk uit te voeren zijn. Een van de methoden is het aanbieden van producten die vaak samen met het geselecteerde artikel gekocht wordt. Een elektronica internetwinkel kan bijvoorbeeld bijpassende inkt aanbieden bij een gekozen printer. Een andere aanpak is het tonen van alternatieven van het gekozen product die vaak duurder zijn. Als voorbeeld worden er gelijkaardige sweaters aangeboden wanneer een sweater op een mode webshop bekeken wordt. Hiernaast zijn nog tal van manieren om het koopgedrag van klanten te beïnvloeden. 

Bovenstaande technieken toepassen in een fysieke handelszaak is minder eenvoudig. Alleen verkopers kunnen verwante of bijhorende artikels aanbieden. Echter wordt niet elke bezoeker geholpen door een personeelslid. Terugkerende klanten zijn vertrouwd met de winkelomgeving en weten het nodige product te vinden, daarom is het inschakelen van een bediende overbodig. Anderzijds is er ook cliënteel die liever op zelfstandige basis winkelt. Hierdoor is het vaak ingewikkelder om in te spelen om het aankoopgedrag van consumenten.

\section{Onderzoeksvraag}
\label{sec:onderzoeksvraag}

Dit onderzoek zal drie machine learning frameworks vergelijken waarmee een afbeeldingclassificatie model ontwikkeld kan worden, oplopend in complexiteit van implementatie. Het zal vergelijken of een complex framework even nauwkeurig is als een eenvoudig toepasbaar framework, met een beperkte en realistische dataset. Daarnaast gaat dit onderzoek na of productherkenning door middel van machine learning een oplossing kan bieden op het beperkt aanbieden van productinformatie in een fysieke winkel. Een praktische applicatie zal verdere productinformatie geven voor een bepaald product, en hierop wordt gecontroleerd of het koopgedrag van de consument beïnvloed wordt. Maar ook of de consument deze applicatie zelf als meerwaarde ziet. Concreet bestaat het onderzoek uit drie deelonderzoeksvragen:
\begin{itemize}
  \item Wat is het nauwkeurigste machine learning framework voor een aangepast afbeeldingclassificatie model met een beperkte dataset?
  \item Wordt een productherkennde applicatie als hulpmiddel beschouwd?
  \item Beïnvloed een productherkennende applicatie het koopgedrag van de consument?
\end{itemize}

%\section{Onderzoeksdoelstelling}
%\label{sec:onderzoeksdoelstelling}
%Wat is het beoogde resultaat van je bachelorproef? Wat zijn de criteria voor succes? Beschrijf die zo concreet mogelijk.

\section{Opzet van deze bachelorproef}
\label{sec:opzet-bachelorproef}

% Het is gebruikelijk aan het einde van de inleiding een overzicht te
% geven van de opbouw van de rest van de tekst. Deze sectie bevat al een aanzet
% die je kan aanvullen/aanpassen in functie van je eigen tekst.

De rest van deze bachelorproef is als volgt opgebouwd:

In Hoofdstuk~\ref{ch:stand-van-zaken} wordt een overricht gegeven van de stand van zaken binnen machine learning, op basis van een grondige literatuurstudie. 

In Hoofdstuk~\ref{ch:methodologie} wordt de structuur van het onderzoek omschreven, alsook de gebruikte frameworks voor het onderzoek.

In Hoofdstuk~\ref{ch:Voorbereiding op onderzoek} wordt uitgelegd wat er allemaal nodig is alvorens een vergelijkend onderzoek gedaan kan worden. 

In Hoofdstuk~\ref{ch:vergelijkend onderzoek} wordt in detail uitgelegd hoe de frameworks getraind zijn en worden de resultaten van het onderzoek vergeleken.

In Hoofdstuk~\ref{ch:Praktisch onderzoek} wordt toegelicht hoe het praktisch onderzoek is verlopen.

%In Hoofdstuk~\ref{ch:stand-van-zaken} wordt een overzicht gegeven van de stand van zaken binnen het onderzoeksdomein, op basis van een literatuurstudie.
%In Hoofdstuk~\ref{ch:methodologie} wordt de methodologie toegelicht en worden de gebruikte onderzoekstechnieken besproken om een antwoord te kunnen formuleren op de onderzoeksvragen.
% TODO: Vul hier aan voor je eigen hoofstukken, één of twee zinnen per hoofdstuk
In Hoofdstuk~\ref{ch:conclusie}, tenslotte, wordt de conclusie gegeven en een antwoord geformuleerd op de onderzoeksvragen. Daarbij wordt ook een aanzet gegeven voor toekomstig onderzoek binnen dit domein.



