%%=============================================================================
%% Voorwoord
%%=============================================================================

\chapter*{Woord vooraf}
\label{ch:voorwoord}

%% TODO:
%% Het voorwoord is het enige deel van de bachelorproef waar je vanuit je
%% eigen standpunt (``ik-vorm'') mag schrijven. Je kan hier bv. motiveren
%% waarom jij het onderwerp wil bespreken.
%% Vergeet ook niet te bedanken wie je geholpen/gesteund/... heeft

In het laatste semester Toegepaste Informatica aan de Hogeschool Gent wordt een scriptie verwacht die relevant is voor deze opleiding.  Aangezien machine learning meer en meer wordt toegepast in allerhande sectoren leek het mij zeer interessant om hierover een bachelorproef te schrijven.  Daarnaast ben ik al sinds kleins af aan gepassioneerd door de winkel van mijn vader, het is een droom om deze zaak later verder te zetten. Ik probeer nu al vaak mee te denken naar nieuwe en innovatieve ideeën voor de winkel. Machine learning combineren met een fysieke winkel door middel van productherkenning, leek voor mij de ideale manier om de opleiding Toegepaste Informatica af te sluiten. Ik hoop natuurlijk dat deze bachelorproef ook nuttig is voor andere zaakvoerders die productherkenning willen toepassen in hun handelszaak.

Alleen had ik deze bachelorproef niet kunnen volmaken. Daarom had ik graag enkele mensen willen bedanken. Vooreerst wil ik mijn promotor Ludwig Stroobant bedanken die mij het vertrouwen heeft gegeven om dit onderzoek uit te voeren. Evenzeer wil ik mijn copromotor Philip Smet bedanken die antwoord bood op al mijn nodige vragen. Tevens waardeer ik ook de steun die mijn ouders mij geboden hebben. Ten slotte wil ik mijn vriendin Evelyn Ghys bedanken. Zij heeft mij enorm gesteund en geholpen, niet alleen tijdens deze proef maar gedurende het volledige academiejaar. 
